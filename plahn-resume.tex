% resume.tex
% vim:set ft=tex spell:

\documentclass[10pt,letterpaper]{article}
\usepackage[letterpaper,margin=0.75in]{geometry}
\usepackage[utf8]{inputenc}
\usepackage{mdwlist}
\usepackage[T1]{fontenc}
\usepackage{textcomp}
\usepackage{tgpagella}
\pagestyle{empty}
\setlength{\tabcolsep}{0em}

% indentsection style, used for sections that aren't already in lists
% that need indentation to the level of all text in the document
\newenvironment{indentsection}[1]%
{\begin{list}{}%
	{\setlength{\leftmargin}{#1}}%
	\item[]%
}
{\end{list}}

% opposite of above; bump a section back toward the left margin
\newenvironment{unindentsection}[1]%
{\begin{list}{}%
	{\setlength{\leftmargin}{-0.5#1}}%
	\item[]%
}
{\end{list}}

% format two pieces of text, one left aligned and one right aligned
\newcommand{\headerrow}[2]
{\begin{tabular*}{\linewidth}{l@{\extracolsep{\fill}}r}
	#1 &
	#2 \\
\end{tabular*}}

% make "C++" look pretty when used in text by touching up the plus signs
\newcommand{\CPP}
{C\nolinebreak[4]\hspace{-.05em}\raisebox{.22ex}{\footnotesize\bf ++}}

% and the actual content starts here
\begin{document}

\begin{center}
{\LARGE \textbf{Jordan Plahn}}

880 Plantation Rd.\ \ \textbullet
\ \ Apt.\ 305\ \ \textbullet
\ \ Blacksburg, VA 24060
\\
(804) 662-0999\ \ \textbullet
\ \ jplahn@vt.edu
\end{center}


\hrule
\vspace{-0.4em}
\subsection*{Education}

\begin{itemize}
	\parskip=0.1em

	\item 
	\headerrow
		{\textbf{Virginia Tech}}
		{\textbf{Blacksburg, VA}}
	\\
	\headerrow
		{\emph{College of Engineering, B.S. Computer Science, B.S. Engineering Science \& Mechanics}}
		{\emph{May 2015}}
	\begin{itemize*}
		\item Major GPA: 3.46
		\item Minor in Mathematics
	\end{itemize*}

\end{itemize}


\hrule
\vspace{-0.4em}
\subsection*{Experience}

\begin{itemize}
	\parskip=0.1em

	\item
	\headerrow
		{\textbf{Lockheed Martin}}
		{\textbf{King of Prussia, PA}}
	\\
	\headerrow
		{\emph{Software Engineer Intern}}
		{\emph{Summer 2014}}
	\begin{itemize*}
		\item Created an end-to-end testing suite to increase test coverage of existing web applications from 0\% to 75\% using various JavaScript utilities
		\item Leveraged Apache Cordova to create a hybrid app for iOS based on an existing web application
		\item Performed weekly code reviews for full time employees with Atlassian Stash and participated in two week sprints
	\end{itemize*}

	\item
	\headerrow
		{\textbf{InfernoRed Technology}}
		{\textbf{Blacksburg, VA}}
	\\
	\headerrow
		{\emph{Software Engineer}}
		{\emph{Feburary 2013 - December 2013}}
	\begin{itemize*}
		\item Developed entire GUI for a Lockheed Martin and Intel sanctioned Windows 8 application
		\item Responsible for identifying and fixing 11 bugs, as well as building a jQuery color chooser, in my first month working on Shazapp, a Microsoft endorsed Windows 8 app kick-starter
	\end{itemize*}
\end{itemize}


\hrule
\vspace{-0.4em}
\subsection*{Projects}

\begin{itemize}

	\item 
	\headerrow
		{\textbf {A Happy Haskell}}
		{Summer 2014}
	\begin{itemize*}
		\item Introductory book aimed at people new to Haskell
		\item Provides ample explanation, examples, and exercises to help people feel comfortable working with Haskell
	\end{itemize*}
	
	\item 
	\headerrow
		{\textbf {MIPS Assembler}}
		{Spring 2014}
	\begin{itemize*}
		\item Assembler capable of translating more than 30 instructions into .txt representations of machine code
		\item Designed project architecture to enable maximum scalability for more instructions and ease of maintenance for changes. \textit{[C, git, vim]} 
	\end{itemize*}
\end{itemize}

\hrule
\vspace{-0.4em}
\subsection*{Core Technical Skills}

\begin{description}
	\item
	\headerrow {Languages:}{}
	\begin{itemize}
		\item \textit{Proficient:} Java, C, \LaTeX
		\item \textit{Working knowledge:} Haskell, JavaScript, HTML, Matlab, Mathematica
	\end{itemize}
	
\end{description}


\hrule
\vspace{-0.4em}
\subsection*{Awards}
\begin{itemize}
	\item \textit{Dr. Robert A. Heller ESM Scholarship (2012 – 2013)}: awarded to a rising junior in ESM with outstanding academic achievement
	\item \textit{ESM Advisory Board Scholarship (2014 – 2015)}: awarded to rising seniors in top 10\% of class
	\item \textit{Computer Science Resources Consortium Scholarship (2014 – 2015)}: merit based scholarship based on academic achievement in computer science
	\item \textit{Phi Kappa Phi National Honor Society (2013 – 2015)}: offers extended to juniors in top 7.5\% of their class
\end{itemize}
\end{document}
